% !TEX program = XeLaTeX
\documentclass[12pt,a4paper,]{article}

% Fonts
\usepackage{libertine}
% This is a nice mathfont, it fits well with Libertine
% Comment the line if you want to use a different one
\usepackage[libertine]{newtxmath}
% The default monofont 
% Again, comment the line below if you want to change it
\usepackage[scaled=.95]{inconsolata}

% Add the following packages to support kableExtra
\usepackage{booktabs}
\usepackage{longtable}
\usepackage{array}
\usepackage{multirow}
\usepackage{wrapfig}
\usepackage{colortbl}
\usepackage{pdflscape}
\usepackage{tabu}
\usepackage{threeparttable}
\usepackage{threeparttablex}
\usepackage[normalem]{ulem}
\usepackage{makecell}
\usepackage{etoolbox}
\usepackage{tocloft}

% Dots in toc
\renewcommand{\cftsecleader}{\cftdotfill{\cftdotsep}}

% Colours
\usepackage[usenames,dvipsnames]{xcolor}
\definecolor{darkblue}{rgb}{0.0,0.0,0.55}

% Spacing
\usepackage{setspace}

% Margin
\usepackage[margin=2cm]{geometry}

% Packages I've been using for different reasons...
\usepackage[backref,pagebackref]{hyperref}
\usepackage{dcolumn}
\usepackage{graphicx}
\usepackage{float}
\floatplacement{figure}{H}
\usepackage{pgf}
\usepackage{tikz}
\usetikzlibrary{arrows}
\usetikzlibrary{positioning}
\usepackage{mathtools}
\usepackage{caption}

% UK English
\usepackage[UKenglish]{babel}
\usepackage[UKenglish]{isodate}
\cleanlookdateon

% Penalties
\exhyphenpenalty=1000
\hyphenpenalty=1000
\widowpenalty=1000
\clubpenalty=1000

% Back references
\renewcommand*{\backref}[1]{}
\renewcommand*{\backrefalt}[4]{%
    \ifcase #1 (Not cited.)%
    \or        Cited on page~#2.%
    \else      Cited on pages~#2.%
    \fi}
\renewcommand{\backreftwosep}{ and~} 
\renewcommand{\backreflastsep}{ and~}

% Hypersetup
\hypersetup{
  linkcolor=Mahogany,
  citecolor=Mahogany,
  urlcolor=darkblue, 
  breaklinks=true, 
  colorlinks=true,
      pdfauthor={Danilo Freire; Manoel Galdino; Umberto Mignozzetti; Jessica Voigt},
      pdfkeywords={accountability, Brazil, field experiment, impact evaluation, state
capacity, technology},
  }

% If XeTex, LuaLaTeX, etc
\usepackage{ifxetex,ifluatex}
\usepackage{fixltx2e} % provides \textsubscript
\ifnum 0\ifxetex 1\fi\ifluatex 1\fi=0 % if pdftex
  \usepackage[T1]{fontenc}
  \usepackage[utf8]{inputenc}
\else % if luatex or xelatex
  \ifxetex
    \usepackage{amssymb,amsmath}
    \usepackage{mathspec}
  \else
    \usepackage{fontspec}
  \fi
  \defaultfontfeatures{Ligatures=TeX,Scale=MatchLowercase}
\fi
% use upquote if available, for straight quotes in verbatim environments
\IfFileExists{upquote.sty}{\usepackage{upquote}}{}
% use microtype if available
\IfFileExists{microtype.sty}{%
\usepackage{microtype}
\UseMicrotypeSet[protrusion]{basicmath} % disable protrusion for tt fonts
}{}

% Language

% Bibliography
\usepackage{natbib}
\bibliographystyle{apalike}
\makeatletter
% Remove comma after author
\setcitestyle{aysep={}}
\patchcmd{\NAT@citex}
	  {\@citea\NAT@hyper@{%
		 \NAT@nmfmt{\NAT@nm}%
		 \hyper@natlinkbreak{\NAT@aysep\NAT@spacechar}{\@citeb\@extra@b@citeb}%
		 \NAT@date}}
	  {\@citea\NAT@nmfmt{\NAT@nm}%
	   \NAT@aysep\NAT@spacechar\NAT@hyper@{\NAT@date}}{}{}
	\patchcmd{\NAT@citex}
	  {\@citea\NAT@hyper@{%
		 \NAT@nmfmt{\NAT@nm}%
		 \hyper@natlinkbreak{\NAT@spacechar\NAT@@open\if*#1*\else#1\NAT@spacechar\fi}%
		   {\@citeb\@extra@b@citeb}%
		 \NAT@date}}
	  {\@citea\NAT@nmfmt{\NAT@nm}%
	   \NAT@spacechar\NAT@@open\if*#1*\else#1\NAT@spacechar\fi\NAT@hyper@{\NAT@date}}
	  {}{}
% Patch case where name and year are separated by aysep
\patchcmd{\NAT@citex}
  {\@citea\NAT@hyper@{%
     \NAT@nmfmt{\NAT@nm}%
     \hyper@natlinkbreak{\NAT@aysep\NAT@spacechar}{\@citeb\@extra@b@citeb}%
     \NAT@date}}
  {\@citea\NAT@nmfmt{\NAT@nm}%
   \NAT@aysep\NAT@spacechar\NAT@hyper@{\NAT@date}}{}{}
% Patch case where name and year are separated by opening bracket
\patchcmd{\NAT@citex}
  {\@citea\NAT@hyper@{%
     \NAT@nmfmt{\NAT@nm}%
     \hyper@natlinkbreak{\NAT@spacechar\NAT@@open\if*#1*\else#1\NAT@spacechar\fi}%
       {\@citeb\@extra@b@citeb}%
     \NAT@date}}
  {\@citea\NAT@nmfmt{\NAT@nm}%
   \NAT@spacechar\NAT@@open\if*#1*\else#1\NAT@spacechar\fi\NAT@hyper@{\NAT@date}}
  {}{}
\makeatother

% Listings

% Verbatim

% Tables

% Graphics

% Make links footnotes instead of hotlinks:
%  \setlength{\emergencystretch}{3em}  % prevent overfull lines
 \providecommand{\tightlist}{%
   \setlength{\itemsep}{0pt}\setlength{\parskip}{0pt}}
   
% Numbered sections
\setcounter{secnumdepth}{5}
% % % Redefines (sub)paragraphs to behave more like sections
% \ifx\paragraph\undefined\else
% \let\oldparagraph\paragraph
% \renewcommand{\paragraph}[1]{\oldparagraph{#1}\mbox{}}
% \fi
% \ifx\subparagraph\undefined\else
% \let\oldsubparagraph\subparagraph
% \renewcommand{\subparagraph}[1]{\oldsubparagraph{#1}\mbox{}}
% \fi
% 
% Spacing
\doublespacing

% Title
\title{The Pitfalls of Bottom-Up Accountability: Evidence from Brazil\thanks{We thank Guilherme Fasolin, Stephen Herzog, and David Skarbek for their
helpful comments. We are specially grateful to the staff of
Transparência Brasil for their excellent research and technical
assistance. Replication data and code are available at
\url{http://github.com/umbertomignozzetti/tdp-accountability}.}}

% Author
\author{Danilo Freire\footnote{Postdoctoral Research Associate, The Political
  Theory Project, Brown University, Providence, RI 02912, USA,
  \href{mailto:danilofreire@brown.edu}{\nolinkurl{danilofreire@brown.edu}},
  \url{http://danilofreire.github.io}.} \and Manoel Galdino\footnote{Executive Diretor, Transparência Brazil, São
  Paulo, SP, Brazil,
  \href{mailto:mcz.fea@gmail.com}{\nolinkurl{mcz.fea@gmail.com}},
  \url{https://www.transparencia.org.br}.} \and Umberto Mignozzetti\footnote{School of International Relations, Fundação
  Getulio Vargas, São Paulo, SP, Brazil and Wilf Family Department of
  Politics, NYU, NY, USA,
  \href{mailto:umberto.mig@nyu.edu}{\nolinkurl{umberto.mig@nyu.edu}},
  \url{http://umbertomig.com}. Corresponding author.} \and Jessica Voigt\footnote{Data Scientist, Transparência Brasil, São Paulo,
  SP, Brazil,
  \href{mailto:voigt.jessica@gmail.com}{\nolinkurl{voigt.jessica@gmail.com}},
  \url{https://www.linkedin.com/in/voigtjessica}.}}

% Date
\date{\today}

% Begin document
\begin{document}
\maketitle

% Abstract
\begin{abstract}
\doublespacing \noindent We study the effect of a mobile phone application designed to increase
citizens' ability to monitor public works in Brazil. The app allowed
individuals to find delayed school constructions in XXX Brazilian cities
and send anonymous messages to politicians requestions information about
construction delays and expected completion times. Our results show that
the app has a null impact on the completion times subsequently reported
by the municipalities. Additionally, we find that few politicians
reacted to citizens' requests. These findings suggest it is difficult to
motivate bottom-up accountability in new democracies, especially when
politicians are unresponsive to non-electoral pressures. This paper has
implications for the design of bottom-up accountability mechanisms.
\vspace{.25cm}

\noindent \textbf{Keywords}: accountability, Brazil, field experiment, impact evaluation, state
capacity, technology
\vspace{.25cm}

\end{abstract}


% Table of Contents
\newpage

\hypertarget{introduction}{%
\section{\texorpdfstring{Introduction\label{sec:intro}}{Introduction}}\label{introduction}}

One of the key determinants of efficient public service provision is a
robust accountability system
\citep{besley2003incentives,cameron2004public,o1998horizontal}. In its
standard definition, accountability is the process of holding
authorities into account for their actions
\citep{finer1941administrative,mulgan2000accountability,o1990bureaucratic}.
Recent studies show that accountability ensures that politicians act on
behalf of voters \citep{freire2010ngp, moncrieffe1998reconceptualizing},
reduces the opportunities for rent-seeking and corruption
\citep{deininger2005does,wenar2006accountability} and increases the
efficiency of the public sector \citep{adsera2003you,bjorkman2009power}.
Moreover, greater accountability is associated with higher levels of
economic growth, not only because it limits state discretion in the
economy, but also because it promotes social cohesion and long-term
investments in human capital
\citep{benhabib2010economic, cox2018executive, suebvises2018social,ponzetto2018social}.

However, accountability systems can take many forms. One promising model
for democratic oversight is \emph{bottom-up accountability}, in which
citizens receive information about the shortcomings of a given project
to help them monitor and pressure underperforming public officials
\citep{kosack2014does, molina2016community,raffler2018weakness}.
Proponents suggest bottom-up is an effective method of accountability
because 1) constituents have first-hand information about the outcomes
of local policies; 2) citizens have incentives to avoid corruption that
directly affects them; 3) policy-makers are particularly sensitive to
social punishment from their own communities
\citep[570]{serra2011combining}. In a nutshell, bottom-up accountability
offers a potential solution to the principal-agent dilemma in public
service by aligning the interests of politicians and bureaucrats with
those of the constituency they serve \citep[2]{raffler2018weakness}.

Here we assess the impact of a technology-driven solution that lowers
the costs of evaluating public works and punishing political
representatives in Brazil. The \emph{Tá de Pé} (TDP) mobile phone
application allows Brazilian citizens to go to school construction
sites, check their status, and request information to public
authorities. TDP users can also take pictures from delayed works and
submit them to a group of trained engineers, who then evaluate the
execution of the project. If the engineers classify the construction as
delayed, TDP sends a message to the mayor's office asking for completion
estimates and explanations about why the building is unfinished. TDP
received the 2016 Google Social Impact grant, earning more than 200
thousand popular votes, and it has been online since April 2017 in 1030
Brazilian municipalities.

Contrary to our expectations, the TDP app has delivered mixed results.
Two findings are positive and in line with our theoretical predictions.
From August 2017 to February 2019, TDP increased the likelihood of
school completion by 6.8 percent (SE = X.X, t-test = XXX). Additionally,
we find that pressure over the bureaucracy increased the investments in
the schools by 157.5 percent (SE = X.XX, t-test = XXX). In contrast, our
results show the app had a null impact on school constructions
statistics. Brazilian politicians did not respond to the citizen's
requests placed via the application. Therefore, our findings raise
questions about the ability of citizens to hold representatives
accountable using bottom-up strategies.

\hypertarget{the-ta-de-pe-app}{%
\section{\texorpdfstring{The \emph{Tá de Pé}
App}{The Tá de Pé App}}\label{the-ta-de-pe-app}}

\hypertarget{methods}{%
\section{Methods}\label{methods}}

\hypertarget{estimation}{%
\subsection{Estimation}\label{estimation}}

Our regression model is specified as follows:

\[Y_{i} = \alpha + \beta T_{i} + \gamma X_{i} + \theta Z_i + \epsilon_{i} \]

Where \(i\) indexes the case. \(Y_i\) in one the outcomes we described
in the previous subsection. The coefficient for \(\beta\) is the
quantity of interest (Average Treatment Effect). \(T_i\) is a binary
treatment indicator in which zero denotes that \(i\) is in the control
group, while 1 indicates that \(i\) received the treatment. \(\gamma\)
is a vector of fixed effects and \(X_i\) is a matrix of 27 Brazilian
states' fixed effects. \(\theta\) is a vector of additional controls and
\(Z_i\) is an array of control variables for the case \(i\). Lastly,
\(\epsilon_i\) is the error term.

We cluster the standard errors at the municipal level, as the investment
decisions are taken by the mayor's offices. As robustness, we fit two
extra models. First, we re-run the main model where we have the
following: (i) we ran a model without controls or fixed effects; (ii) we
run the models adding the control variables, which are the same as the
ones used in the covariate balance tests; (iii) we run the regression
model with State fixed effects. The results are in the Appendix. Second,
we run the main models using inverse probability weights where we used
block randomization and a one-to-one nearest neighborhood matching. We
use the control variables as matching characteristics. These robustness
checks are intended to correct the imbalance caused by the minimal
control group approach that we employed in all three interventions.

\hypertarget{results}{%
\section{Results}\label{results}}

\hypertarget{discussion}{%
\section{Discussion}\label{discussion}}

\hypertarget{conclusion}{%
\section{Conclusion}\label{conclusion}}

\newpage
\setlength{\parindent}{0cm}
\setlength{\parskip}{5pt}

% More bibliography
\bibliography{references.bib}

\end{document}
