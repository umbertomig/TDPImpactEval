% !TEX program = XeLaTeX
\documentclass[12pt,a4paper,]{article}

% Fonts
\usepackage{libertine}
% This is a nice mathfont, it fits well with Libertine
% Comment the line if you want to use a different one
\usepackage[libertine]{newtxmath}
% The default monofont 
% Again, comment the line below if you want to change it
\usepackage[scaled=.95]{inconsolata}

% Add the following packages to support kableExtra
\usepackage[normalem]{ulem}
\usepackage{array}
\usepackage{booktabs}
\usepackage{colortbl}
\usepackage{etoolbox}
\usepackage{longtable}
\usepackage{makecell}
\usepackage{multirow}
\usepackage{pdflscape}
\usepackage{subcaption}
\usepackage{tabu}
\usepackage{threeparttablex}
\usepackage{threeparttable}
\usepackage{tocloft}
\usepackage{wrapfig}

% Dots in toc
\renewcommand{\cftsecleader}{\cftdotfill{\cftdotsep}}

% Colours
\usepackage[usenames,dvipsnames]{xcolor}
\definecolor{darkblue}{rgb}{0.0,0.0,0.55}

% Spacing
\usepackage{setspace}

% Margin
\usepackage[margin=2cm]{geometry}

% Packages I've been using for different reasons...
\usepackage{hyperref}
\usepackage{dcolumn}
\usepackage{graphicx}
\usepackage{float}
\floatplacement{figure}{H}
\usepackage{pgf}
\usepackage{mathtools}
\usepackage{caption}

% UK English
\usepackage[UKenglish]{babel}
\usepackage[UKenglish]{isodate}
\cleanlookdateon

% Penalties
\exhyphenpenalty=1000
\hyphenpenalty=1000
\widowpenalty=1000
\clubpenalty=1000

% Hypersetup
\hypersetup{
  linkcolor=Mahogany,
  citecolor=Mahogany,
  urlcolor=darkblue, 
  breaklinks=true, 
  colorlinks=true,
      }

% If XeTex, LuaLaTeX, etc
\usepackage{ifxetex,ifluatex}
\usepackage{fixltx2e} % provides \textsubscript
\ifnum 0\ifxetex 1\fi\ifluatex 1\fi=0 % if pdftex
  \usepackage[T1]{fontenc}
  \usepackage[utf8]{inputenc}
\else % if luatex or xelatex
  \ifxetex
    \usepackage{amssymb,amsmath}
    \usepackage{mathspec}
  \else
    \usepackage{fontspec}
  \fi
  \defaultfontfeatures{Ligatures=TeX,Scale=MatchLowercase}
\fi
% use upquote if available, for straight quotes in verbatim environments
\IfFileExists{upquote.sty}{\usepackage{upquote}}{}
% use microtype if available
\IfFileExists{microtype.sty}{%
\usepackage{microtype}
\UseMicrotypeSet[protrusion]{basicmath} % disable protrusion for tt fonts
}{}

% Language

% Bibliography
\usepackage{natbib}
\bibliographystyle{apalike}
\makeatletter
% Remove comma after author
\setcitestyle{aysep={}}
\patchcmd{\NAT@citex}
	  {\@citea\NAT@hyper@{%
		 \NAT@nmfmt{\NAT@nm}%
		 \hyper@natlinkbreak{\NAT@aysep\NAT@spacechar}{\@citeb\@extra@b@citeb}%
		 \NAT@date}}
	  {\@citea\NAT@nmfmt{\NAT@nm}%
	   \NAT@aysep\NAT@spacechar\NAT@hyper@{\NAT@date}}{}{}
	\patchcmd{\NAT@citex}
	  {\@citea\NAT@hyper@{%
		 \NAT@nmfmt{\NAT@nm}%
		 \hyper@natlinkbreak{\NAT@spacechar\NAT@@open\if*#1*\else#1\NAT@spacechar\fi}%
		   {\@citeb\@extra@b@citeb}%
		 \NAT@date}}
	  {\@citea\NAT@nmfmt{\NAT@nm}%
	   \NAT@spacechar\NAT@@open\if*#1*\else#1\NAT@spacechar\fi\NAT@hyper@{\NAT@date}}
	  {}{}
% Patch case where name and year are separated by aysep
\patchcmd{\NAT@citex}
  {\@citea\NAT@hyper@{%
     \NAT@nmfmt{\NAT@nm}%
     \hyper@natlinkbreak{\NAT@aysep\NAT@spacechar}{\@citeb\@extra@b@citeb}%
     \NAT@date}}
  {\@citea\NAT@nmfmt{\NAT@nm}%
   \NAT@aysep\NAT@spacechar\NAT@hyper@{\NAT@date}}{}{}
% Patch case where name and year are separated by opening bracket
\patchcmd{\NAT@citex}
  {\@citea\NAT@hyper@{%
     \NAT@nmfmt{\NAT@nm}%
     \hyper@natlinkbreak{\NAT@spacechar\NAT@@open\if*#1*\else#1\NAT@spacechar\fi}%
       {\@citeb\@extra@b@citeb}%
     \NAT@date}}
  {\@citea\NAT@nmfmt{\NAT@nm}%
   \NAT@spacechar\NAT@@open\if*#1*\else#1\NAT@spacechar\fi\NAT@hyper@{\NAT@date}}
  {}{}
\makeatother

% Listings

% Verbatim

% Tables

% Graphics

% Make links footnotes instead of hotlinks:
%  \setlength{\emergencystretch}{3em}  % prevent overfull lines
 \providecommand{\tightlist}{%
   \setlength{\itemsep}{0pt}\setlength{\parskip}{0pt}}
   
% Numbered sections
\setcounter{secnumdepth}{0}
% % % Redefines (sub)paragraphs to behave more like sections
% \ifx\paragraph\undefined\else
% \let\oldparagraph\paragraph
% \renewcommand{\paragraph}[1]{\oldparagraph{#1}\mbox{}}
% \fi
% \ifx\subparagraph\undefined\else
% \let\oldsubparagraph\subparagraph
% \renewcommand{\subparagraph}[1]{\oldsubparagraph{#1}\mbox{}}
% \fi
% 
% Spacing
\doublespacing

% Title

% Author

% Date
\date{}

% Begin document
\begin{document}

% Abstract


% Table of Contents
\noindent Research \& Politics

\vspace{.1cm}

\noindent 17 December 2019

\vspace{.5cm}

\noindent Dear Editor and Reviewer,

\vspace{.5cm}

\noindent We would like to thank you for the opportunity to revise our
manuscript, ``Bottom-Up Accountability and Public Service Provision:
Evidence from a Field Experiment in Brazil'' (Manuscript ID
RAP-19-0144). We have made many revisions to the paper along the lines
suggested by the editor and the anonymous reviewer and we believe the
manuscript has improved significantly as a result.

We have worked specially hard to clarify the connection between
information provision and mobilisation, a point emphasised by both the
editor and the reviewer. Our efforts in that regard include, first, a
new section in the SI Appendix with a detailed description of \emph{Tá
de Pé}'s functionalities and the associated Facebook campaigns. We
explain the precautions we have taken to ensure that we correctly
identify the effect of the TDP app on school outcomes. Second, we
provide a thorough discussion of our additional data sources. More
specifically, we show how the Brazilian Ministry of Education collects
and publicises their data and indicate how our experiment uses that
information to evaluate the robustness of our findings. We believe the
new sections help give the results a clearer interpretation and
strenghten our main arguments.

Related, we have included a discussion regarding app usage as suggested
by the reviewer. We have included information provided by Google
Analytics not only on app downloads but also on app usage, which
addresses another important concern raised by the reviewer. The data
show that the app usage was significant, yet it was not enough to
trigger any responses from local governments.

Lastly, we have rewritten some parts of the main text to highlight that
our experiment hinges on participation only, and that the effect of
mobilisation and pressure on school construction outcomes require
further analyses. We believe our paper now reflects our results more
accurately.

Below we discuss in more detail these and other changes we have made to
the manuscript in response to comments from the editor and the reviewer.
We thank the editor again for the opportunity to revise the manuscript,
and the reviewer for his/her extremely helpful comments.

\vspace{.5cm}

\noindent Sincerely,

\vspace{.5cm}

\noindent The Authors

\newpage

\hypertarget{editor-comments-and-responses}{%
\section{Editor Comments and
Responses}\label{editor-comments-and-responses}}

\textbf{1)} The editor writes: ``\emph{The reviewer confirmed my own
sense that the manuscript and underlying field experiment are
interesting, although you will see that the reviewer suggests many ways
the manuscript can be improved. I am in full agreement with the reviewer
that you need to reflect more deeply about the theory and evidence for
why information provision via the app could be connected to
mobilization, -- otherwise the null result is not interesting. As the
reviewer notes, the null result could be due to a disconnect between
information provision and action, or due to who selects into downloading
the app.}''

\vspace{.5cm}

\noindent \textbf{Response:}

\begin{quotation}
We agree with the editor and the reviewer in regards to the link between
information provision and mobilisation. We believe the main issue lies in the
lack of evidence for mobilisation, that is, information on how users interacted
with the application and whether they have used the app consistently. In the
revised version of the manuscript, we provide data from Google Analytics which
indicate that users did engage with the app. According to the service, the app
had 6,092 active users in intervention 1 and gained 4,078 new users during
intervention 2. On average, each user launched 60 sessions per app install, an
indication of their actual engagement with the application. Moreover, the data
also show that the app had more than 53,000 screen visualisations, with an
average of 2.42 visualisations per user. We have added this information to the
main text and the online appendix. The additions are marked in \textbf{boldface}.
We elaborate further on this issue in the response to Reviewer 1 below.
\end{quotation}

\hypertarget{reviewer-1-comments-and-responses}{%
\section{Reviewer 1 Comments and
Responses}\label{reviewer-1-comments-and-responses}}

\textbf{1)} Reviewer 1 writes: ``\emph{My biggest concern is there is no
discussion of the actual bottom-up accountability that may (or may not)
have occurred. {[}\ldots{}{]} The authors have devised a nice experiment
and I believe their design to have been appropriately administrated. But
the design hinges on several critical mechanisms that go largely
undiscussed: not only must information be available to citizens such
that they might participate in local oversight, but said citizens must
1) access the information and 2) act on said information. This is
mentioned in the Discussion---that perhaps the app was insufficient to
spur collective action---but that also seems like an empirical question
the authors could feasibly address. {[}\ldots{}{]} Once people
downloaded the App, did they use it?}''

\vspace{.5cm}

\noindent \textbf{Response:}

\begin{quotation}
This first point raised by the reviewer is connected to the response we have
provided above. We have made several changes to the main paper and the appendix
to address this important concern. First, we have added a the following section
to the appendix in order to clarify how the TDP app works:

\vspace{.5cm}

3. Treatment Definition and Mechanism

The \textit{Tá de Pé} app

1. Users download the app
2. Use the app and report the schools to the mayors' office
3. Our team took the reported schools and started implementing the pressure
on the mayors' office.
4. Should the pressure have work, we would witness improvements on governmental
     data regarding the school construction progress. 

For the step (4), we relied on monthly issued administrative data by the
Brazilian Ministry of Education. The Ministry of Education provides monthly
data on the school constructions that are being build with Federal resources,
and that are delayed across the country. To compute the delay we use the
finishing dates reported by the municipality at the time they sign the
agreement with the Federal Government. This data helped us to keep the App
updated and the outcomes in the paper are selected to be six months before
(pre-treatment) and after the intervention. This data was constantly updated by
the Ministry of Education, and we found no red flags in the overall quality of
the data.

Therefore, steps (1), (3), and (4) were standardized to not vary across the
country. The only source of variation in our intervention should come from step
(2).

In step (2), we applied the treatment by removing schools within municipalities
out of the app. In the first intervention, that happened at the municipal
level, we excluded all schools within a given randomly selected municipality.
In the second intervention, we excluded randomly selected schools out of the
App. Removing a school means that the school was in the dataset provided by the
Brazilian Ministry of Education, but it did not show up in the App. Therefore,
our treatment effect represents the school being present (treatment) or absent
(control) in the App.

In terms of what this represents, we argue that the App is promoting two
crucial features for bottom-up accountability: first, it is facilitating people
to connect with information that would be otherwise be in an obscure
spreadsheet at the Ministry of Education website; second, whenever people find
this information troublesome, we are providing an easy way to them to act upon
the information. As the design was intuitive and straightforward, we believe
that most of Brazilians would find it easy to use the App.

Moreover, the treatment to be successful does not require that everyone in the
municipality uses the App. Clearly that more usage means more pictures and
denounces, and this represents more pressure over the mayors office, what could
revert into a higher effect. However, just one denounce would be enough, as it
can be directed to the Brazilian General Comptroller (CGU), and this could
potentially deny future funding to the municipality.




\end{quotation}

\newpage
\setlength{\parindent}{0cm}
\setlength{\parskip}{5pt}

% More bibliography
\bibliography{references.bib}

\end{document}
